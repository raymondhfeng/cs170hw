\begin{qunlist}
\q{$\bigstar$}{Course Syllabus}

Before you answer any of the following questions, please read over the syllabus carefully. The syllabus is pinned on the Piazza site. For each statement below, write \textit{OK} if it is allowed by the course policies and \textit{Not OK} otherwise.

\begin{enumerate}[(a)]
\item You ask a friend who took CS 170 previously for her homework solutions, some of which overlap with this semester's problem sets. You look at her solutions, then later write them down in your own words.
\begin{mdframed}
	\textbf{Solution} Not OK
\end{mdframed}

\item You had 5 midterms on the same day and are behind on your homework. You decide to ask your classmate, who's already done the homework, for help. He tells you how to do the first three problems.
\begin{mdframed}
	\textbf{Solution} Not OK
\end{mdframed}

\item You look up a problem online to search an algorithm, write it in your words and cite the source.
\begin{mdframed}
	\textbf{Solution} Not OK
\end{mdframed}

\item You were looking up Dijkstra's on the internet, and run into a website with a problem very similar to one on your homework. You read it, including the solution, and then you close the website, write up your solution, and cite the website URL in your homework writeup.
\begin{mdframed}
	\textbf{Solution} OK
\end{mdframed}

\item You are working on the homework in one of the TA's office hours with other people. You hear that student John Doe asked the TA if his solution is correct or not and the TA is explaining it. You join their conversation to understand what John has done.
\begin{mdframed}
	\textbf{Solution} OK
\end{mdframed}

\end{enumerate}

\q{$\bigstar\bigstar$}{Asymptotic Complexity Comparisons}

\begin{enumerate}[(a)]
\item
Order the following functions so that $f_i \in O(f_j) \iff i \le j$. Do not
justify your answers.
\begin{enumerate}[(i)]
\item $f_1(n) = 3^{n}$
\item $f_2(n) = n^{1 \over 3}$
\item $f_3(n) = 12$
\item $f_4(n) = 2^{\log_2 n}$
\item $f_5(n) = \sqrt{n}$
\item $f_6(n) = 2^n$
\item $f_7(n) = \log_2 n$
\item $f_8(n) = 2^{\sqrt n}$
\item $f_9(n) = n^3$
\end{enumerate}
\begin{mdframed}
	\textbf{Solution} 
	$$O(12) \in O(\log_2(n)) \in O(2^{\log_2(n)}) \in O(n^{\frac{1}{3}}) \in O(n^{\frac{1}{2}}) \in O(n^3) \in O(2^{\sqrt{n}}) \in O(2^n) \in O(3^n)$$
	$$3,7,4,2,5,9,8,6,1$$
\end{mdframed}


\item
In each of the following, indicate whether $f = O(g)$, $f = \Omega(g)$, or both (in which case $f = \Theta(g)$). \textbf{\textit{Briefly}} justify each of your answers.

$\begin{tabu}{r l l}
    & f(n) & g(n) \\
(i) & \log_{3} n & \log_{4} n \\
(ii) & n \log (n^4) & n^2 \log (n^3) \\
(iii) & \sqrt[]{n} & (\log n)^3 \\
(iv) & 2^n & 2^{n+1} \\
(v) & n & (\log n)^{\log\log n} \\
(vi) & n + \log n & n + (\log n)^2 \\
(vii) & \log n! & n \log n \\
\end{tabu}$

\answer{\\
i. $$\frac{\ln(n)}{\ln(3)} \in \theta(\frac{\ln(n)}{\ln(4)}) \Rightarrow f \in \theta(g)$$
ii. $$\lim_{n\to\infty}\frac{4\log(n)}{3n\log(n)}=0$$
$$f \in O(g)$$
iii. $$\lim_{n\to\infty}\frac{\sqrt{n}}{(\log(3n))^3}$$
Take the derivative wrt n of numerator and denominator five times, and L'Hopital gives:
$$\lim_{n\to\infty}\frac{\sqrt{n}}{48} = \infty \Rightarrow f \in \Omega(g)$$
iv. $$\forall n \in \mathbb{R}, a*2^n \leq 2^{n+1} \leq b*2^n, a=\frac{1}{4}, b=4$$
$$f \in \theta(g)$$
v. 
For large n.
$$2\log(\log(\log(n))) \leq \log(\log(n))$$ 
$$\Rightarrow (\log(\log(n)))^2 \leq \log(n)$$
$$\Rightarrow \log((\log(n))^{\log(\log(n))}) \leq \log(n)$$
$$\Rightarrow (\log(n))^{\log(\log(n))} \leq n$$
$$\Rightarrow g \leq f$$
$$f \in \Omega(g)$$
vi. 
I omit the calculus.
$$\lim_{n\to\infty}\frac{f}{g} = 0 \Rightarrow f \in O(g)$$
vii.
$$(\frac{n}{2})^{\frac{n}{2}} \leq n!$$
$$\frac{n}{2}(\log(n) - \log(2)) \leq \log(n!)$$
$$O(n\log(n)) \in O(\log(n!))$$
$$f \in \Omega(g)$$
}

\item Let $f(\cdot)$ be a function. Consider the equality
\[\sum_{i=1}^{n}f(i)\ \in \ \Theta(f(n)),\]
give a function $f_1$ such that the equality holds, and a function $f_2$ such that the equality does not hold.

\answer{
$$\sum_{i=1}^n2^n = 2^{n+1} \in \theta(2^n)$$
$$\sum{i=1}^nn = \frac{n(n+1)}{2} \in \theta(n^2){\tiny }$$
$$f_1(n) = 2^n, f_2(n) = n$$
}

\item Prove or disprove: If $f:\N\to\N$ is any positive-valued function,
then either (1) there exists a constant $c>0$ so that
$f(n) \in O(n^c)$, or (2) there exists a constant $\alpha>1$ so that
$f(n) \in \Omega(\alpha^n)$.

\answer{
Because the function maps natural numbers to natural numbers, there are only a number of possible options for the time complexity of $f$.
$$f(n) \in \theta(1), \theta(n), \theta(n^c), \theta(c^n), \theta(n!), \theta(n^n)$$
Then, all of these possibilities satisfy the claim. 
$$f(n) = 1 \in O(n), f(n) = n \in O(n^2), f(n) = n^k \in O(n^{k}), f(n) = k^{n} \in \Omega(k^n), f(n) = n! \in \Omega(k^n), f(n) = n^n \in \Omega(k^n)$$
}
  
\end{enumerate}


\q{$\bigstar\bigstar\bigstar$}{Recurrence Relations}

Derive an asymptotic {\em tight} bound for the following $T(n)$. Cite any theorem you use.

\begin{enumerate}[(a)]

\item $T(n)=2\cdot T(\frac{n}{2})$ $+$ $\sqrt n$.

\answer{
Using the Master Theorem, we have:
$$a = 2, b = 2, d = \frac{1}{2}$$
$$T(n) \in \theta(n^{log_{b}(a)}) = \theta(n)$$
}

\item $T(n) = T(n-1) + c^n$ for constants $c>0$.

\answer{
Each call of the function on $n$ recursively calls itself on $n-1$, and then does $c^n$ work. The call $T(1)$ returns 1, and so all the work can be seen as the sum of $c^n$ from 2 to $n$. 
$$\sum_{i=2}^{n}c^i=\frac{a_1(1-r^n)}{1-r}=\frac{c^2(1-c^{n-1})}{1-c} \in \theta(c^n)$$   
}

\item $T(n) = 2T(\sqrt{n}) + 3$, and $T(2) = 3$.

\answer{
$$T(n) = 2T(\sqrt{n})+3$$
$$T(n) = 2(2T(\sqrt{\sqrt{n}})+3)+3$$
Because we are taking the square root of $n$ each time, and we stop when $n=2$, we can think of the recursion as halving the exponent each level deeper. Thus, because $n=2^{log_2(n)}$, the number of halvings before the exponent becomes 1 is $log(log(n))$. Thus:
$$T(n) = 2^{log(logn)}T(2)+3*\sum_{i=0}^{log(logn)} \in \theta(log(n))$$
}
 
\end{enumerate}


\q{$\bigstar\bigstar\bigstar\bigstar$}{Recurrence Relations Part II}

Solve the following recurrence relations and give a $\Theta$ bound for each of them.
\begin{enumerate}[(a)]
\item

\begin{enumerate}[(i)]
\item $T(n) = 3T(n/4) + 4n^2$\\
\answer{
Using the Master theorem, where
$$a=3, b=4, d=2, d > \log_b(a)$$
$$T(n) \in \theta(n^2)$$
}
\item $T(n) = 45T(n/3) + .1n^3$\\
\answer{	
Using the Master theorem, where 
$$a=45, b=3, d=3, d < \log_b(a)$$
$$T(n) \in \theta(n^{\log_3(45)})$$
}
\item $T(n) = 2T(\sqrt{n}) + 5$, and $T(2) = 5$. (Hint: this means the recursion tree stops when the problem size is $2$)
\end{enumerate}
\item
\answer{
This is similar to problem 4.a.3, so we can go about it the same way. 
$$T(n) = 2T(\sqrt{n}) + 5$$
$$T(n) = 2((2T(\sqrt{\sqrt{n}} + 5))) + 5$$
Thus, we know from before that there will be a depth of $\log(\log(n)$ in the recursive tree, and so we can express $T(n)$ as
$$T(n) = 2^{\log(\log(n)}T(2) + 5*\sum_{i=0}^{\log(logn))}2^i \in \theta(\log(n))$$
}
\begin{enumerate}[(i)]
\item Consider the recurrence relation $T(n) = 2T(n/2) + n \log n$. We can't plug it directly into the Master theorem, so solve it by adding the size of each layer.
\textit{Hint: split up the $\log (n/(2^i))$ terms into $\log n - \log (2^i)$, and use the formula for arithmetic series.}\\
\answer{
From the recursion tree, we have the summation:
$$\sum_{i=0}^{log_2(n)}2^i\frac{n}{2^i}\log(\frac{n}{2^i})$$
$$=n*\sum_{i=0}^{log_2(n)}(\log(n)-i\log(2))$$
$$=n*\sum_{i=0}^{log_2(n)}\log(n) - n\log(2)\sum_{i=0}^{\log_2(n)}i$$
$$=n(\log(n))^2-n\log(2)\frac{\log_2(n)(\log_2(n) + 1)}{2}$$
$$T(n) \in \theta(n(\log(n))^2)$$

}


\item A more general version of Master theorem, like the one on \href{https://en.wikipedia.org/wiki/Master_theorem}{Wikipedia}, incorporates this result. The case of the master theorem which applies to this problem is:

\textit{If $T(n) = aT(n/b) + f(n)$ where $a \geq 1$, $b > 1$, and $f(n) = \Theta(n^c \log^k n)$ where $c = \log_b a$, then $T(n) = \Theta(n^c \log^{k+1} n)$. }

Use the general Master theorem to solve the following recurrence relation:

$T(n) = 9T(n/3) + n^2 \log ^3 n$.

\answer{
Using the general Master theorem:
$$a=9, b=3, c=2, k=3$$
$$T(n)=\theta(n^2(\log(n))^4)$$
}
\end{enumerate} 
\end{enumerate}

\clearpage

\q{$\bigstar\bigstar\bigstar\bigstar$}{Two Sorted Arrays}

You are given two sorted arrays, each of size $n$. Give as efficient an algorithm as possible to find the $k$-th smallest element in the union of the two arrays. What is the running time of your algorithm as a function of $k$ and $n$?
{\em (You need to give a four-part solution for this problem.)}

\answer{\\ 
\textbf{Main Idea: } In order to take advantage of both arrays being sorted, notice that when looking at the medians of the two arrays, as well as the index desired, then at any step we can eliminate half of one array. Say we have our 2n elements, as well as two arrays A and B. Then there are a number of cases.
\begin{enumerate}
\item 
$k>n$, and median of A less than median of B. We can eliminate the lower half of A. And shift our k back by half of A in order to compensate. This is because if our index was in A, then it would have to be in the upper half. Suppose it was in the lower half. Then that means that there are $>\frac{n}{2}$ elements after it in the upper half of A. Furthermore, because the median of B is larger than that of A, then that means that there would also be $>\frac{n}{2}$ elements after it in the upper half of B. But this is impossible! It cannot be that there are both $>\frac{n}{2}$ elements greater than the kth element in both A and B, because that would mean $k>n$ is false. Thus, it must be valid to remove the lower half of A, as it is impossible for the kth element to be there. 
\item 
$k>n$, and median of A greater than median of B. With similar reasoning, eliminate the lower half of A and shift k back by half of A to compensate.
\item 
$k<n$, and median of A less median of B. With similar reasoning, eliminate the upper half of B. No need to shift k. 
\item 
$k<n$, and median of A greater than median of B. With similar reasoning, eliminate the upper half of A. No need to shift k. 
\end{enumerate}
\textbf{Pseudocode:} \\
kthElement(a, b, k)\\
\tab mid1 = middle element of first array\\
\tab mid2 = middle element of the second array\\
\tab if a has length 0\\
\tab \tab return kth element of b\\
\tab elif b has length 0\\
\tab \tab return kth element of a\\
\tab \tab if k is greater than sum of mid1 and mid 2\\
\tab \tab \tab if a[mid1] greater than b[mid2]\\
\tab \tab \tab \tab return kthElement(a, second half of b, k - mid2 - 1)\\
\tab \tab \tab else \\
\tab \tab \tab \tab return kthElement(second half of a, b, k - mid1 - 1)\\
\tab \tab else\\
\tab \tab \tab if a[mid1] greater than b[mid2]\\
\tab \tab \tab \tab return kthElement(first half of a, b, k)\\
\tab \tab \tab else \\
\tab \tab \tab \tab return kthElement(a, first half of b, k)\\

}

\q{$\bigstar\bigstar\bigstar\bigstar\bigstar$}{Merged Median} \\
Given $k$ sorted arrays of length $l$, design an efficient algorithm to finding the median element of all the $n=kl$ elements. Your algorithm should run  asymptotically faster than $O(n)$. Your answer from 5 may be helpful.

{\em (You need to give a four-part solution for this problem.)}

\answer{}


\end{qunlist}
